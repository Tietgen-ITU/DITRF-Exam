\section{Business Model Design}
Business model performance assessment viser at Nemlig.com's backstage trænger til et løft. Herudover, viser key forces at teknologi trend med at bruge robotter til at sørge for at automatisere en del af driften går frem. Til det er Nemlig.com stadig en af de dyreste dagligvarebutikker, hvilket i en krise gør deres vare endnu dyrere. 

\textit{From High Cost to Low Cost} kan bruges som inspiration til at gøre brug af større automatisering og AI til at optimisere schedulering af pakning, kan man opsige de flere tusinde ressourcer der pakker dag ind og dag ud. Hertil gør at de ikke skal have ekstra løn for aften vagter. Derved reducere Nemlig.com deres omkostninger, som kan bruges til at konkurrerer endnu mere med priserne.

Derfor vil den nye business model se således ud:
\centeredpic[0.45]{NY BMC - Nemlig.png}{En illustration af den nye business model canvas. Noter farvet grøn er nye elementer i blokkende}{nybmc}

\subsection{Assessment Questions}
Assessment Questions af figur \ref{fig:nybmc} vil kunne blive set i \autoref{bilag:ass-ques}. Resultatet viser en forbedret backstage ved at indføre automatisering og indrage in partner til at varetage robotten. Hertil, viser det at Nemlig.com's frontstage forbliver stærk med deres række af value propositions leveret gennem hjemmesiden.  