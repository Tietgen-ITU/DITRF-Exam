\section{Hypoteser \& Test}

\subsection{Assumption mapping}
\centeredpic[0.45]{Assumption mapping.png}{Mapping af hypoteser. Kilder: (\cite{vægter}), (\cite{land-føde-ukraine-pris}), (\cite{andelenergi:timepris}), (\cite{ocado-auto})}{assumpmap}

\subsection{Test \& learnings}
I figur \ref{fig:assumpmap} kan det ses at der er 3 hypoteser, som er vigtige og ikke har nogen evidens. Her indgår 2 feasibility og 1 viability hypotese. Testing ville foregår med følgende metoder i given rækkefølge:
\begin{description}
    \item [Partner \& Supplier Interviews] Vil bruges til at kunne finde nogle partners med forstand på robot teknologi indenfor automatisering af lagerhuse. Hertil vil man teste muligheden for om visionen er mulig ved hjælp af deres response.

    \item [Life-Sized Prototype] gør os i stand til at teste om robotterne er dyrere i drift og om hvor stabilt det er. Der bliver oprettet en mindre test til at se om det kan fungere i praksis inden en eventuel beslutning om at udvide løsningen skal tages.
    
\end{description}

Testing cards med tilhørende learnings cards kan ses i \autoref{appendix:test}.
