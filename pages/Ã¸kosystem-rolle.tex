\section{Økosystemets rolle}
Nemlig.com befinder sig i "køb og salg af varer" i det valgte økosystems supply chain.

Nemlig.com har med sin online platform disrupted frontstage for dagligvarehandel. I figur \ref{fig:cp} kan der ses hvilke pains and gains nemlig.com løser for Børnefamilie customer segmentet.
\centeredpic[0.29]{Customer Profile - Børnefamilier.png}{Overblik over Nemlig.com's value proposition. Kilder: (\cite{dansk-erhverv:dagligvarer-på-nettet}), (\cite{samvirke-fam}), (\cite{landbrug-fødevare:online-handel}), (\cite{nemlig.com:nemlig})}{cp}

\subsection{Business model canvas}
\centeredpic[0.45]{BMC - Nemlig.png}{Nemlig.com's Business Model. Kilder: (\cite{linkedin:nemlig-indpakning}), \cite{nemlig.com:nemlig}}{bmc}

\subsection{Business Model Performance Assessment}
Business model performance assessment kan ses i \autoref{appendix:perf}. Kort fortalt, differentiere nemlig.com sig ikke fra deres konkurrenter i måden de udfører deres aktiviteter. Dette gør at deres produkter forbliver i den dyrere ende af pris-klassen, selvom deres margin burde tillade at sænke priserne.
